
\section{Conclusions}

This paper introduces the Elo-R rating system, which is in part a generalization of the two-player Glicko system, allowing an unbounded number of players. It assumes the players' performances, while potentially hard to quantify directly, can be ranked in a total order. As a natural consequence of some technical modeling assumptions, Elo-R is far more robust to atypical performances than any alternative known to the author.

Applications include many types of sports and video games, as well as programming contests, which presently rely on less rigorously derived models and hacks. The modeling assumptions are best suited to events where the players have minimal targeted interactions against one another, and instead compete individually to score better than rival players in an ongoing array of challenges. For instance, suppose we want to measure a person's skill in traversing obstacle courses, where the course design changes weekly. Completion times are only meaningful on a single course. However, if we treat each course as a match in Elo-R, it becomes possible to quantify and compare the skills of individuals, even if they have never completed the same course together.